
\chapter{Vorgehensweise}

\section{Datengewinnung}

Die \textbf{Trainingsdaten} wurden mit Hilfe des DuckieTown-Simulator erstellt.

\section{Netzwerkarchitektur}

Unsere \textbf{Netzwerkarchitektur} besteht aus \textbf{einer Normalisierungsschicht} (normalization layer), \textbf{fünf Faltunsschichten} (convolutional layers) und \textbf{drei Fully-Connected-Schichten} (fully connected layers). Das Netzwerk nimmt das  Kamerabild des DuckieBots als Eingabe entgegen, wobei das obere drittel des Kamerabildes entfernt wurde. Eine schematsche Darstellung der Netzarchitektur ist in Abbildung \ref{network-architecture} dargestellt. \\

Die \textbf{Normalisierungsschicht} des Netzwerks kümmert sich um die Normalisierung des Eingabebildes, wodruch die GPU-Verabeitung beschleunigt wird. \\

Die \textbf{Faltungsschichten} kümmern sich um die \textbf{Extraktion von Bildmerkmalen}. Die ersten drei Faltungsschichten nutzen dabei einen 5x5 Kernel mit einer 2x2 Schrittbreite (stride) und die letzen beiden einen 3x3 Kernel mit einer 1x1-Schrittbreite. \\

Nach den Faltungsschichten folgen drei \textbf{Fully-Connected-Schichten} die uns schlussendlich die \textbf{geschätzte Distanz} $d$ und \textbf{geschätzte Orientierung} $\Theta$ liefern.


\begin{figure}[H]
	\centering
	\includegraphics[width=0.75\textwidth]{kapitel4/images/network_architecture.png}
	\caption{Schematische Darstellung der Netzwerkarchitektur}
	\label{network-architecture}
	\vspace{0.2cm}
	\quelle\url{https://d3i71xaburhd42.cloudfront.net/0e3cc46583217ec81e87045a4f9ae3478a008227/5-Figure4-1.png}
\end{figure}




