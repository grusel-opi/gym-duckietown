\setcounter{tocdepth}{2}  %% Uebreschriften bis subsectionw ins Inhaltsverzeichnis
\setcounter{secnumdepth}{3}  %% Nummerierung bis subsection


%%% Codebeispiele - Style
\DeclareCaptionFont{white}{\color{white}}
\DeclareCaptionFormat{listing}{\colorbox{gray}{\parbox{\textwidth}{#1#2#3}}}

% Entfernt Kapitel Ueberschrift
% Bsp.
% 	ALT:
%       Kapitel 1
%       Einführung
%
% 	NEU:
% 		1 Einführung
%
\renewcommand*\chapterheadstartvskip{\vspace{-\topskip}}

\setlength\parindent{0pt}
\definecolor{keywordsColor}{rgb}{0.000000, 0.000000, 0.635294}
\definecolor{stringColor}{rgb}{0.558215, 0.000000, 0.135316}

\lstset{
	frame=top,frame=bottom,
	basicstyle=\small\normalfont\sffamily,
	commentstyle=\color{gray}\textit,
	stepnumber=1,                           
	numbersep=2pt,
	tabsize=2,
	extendedchars=true,
	breaklines=true,
	captionpos=t,
	mathescape=true,
	stringstyle=\color{white}\ttfamily,
	showspaces=false,
	showtabs=false,            
	xleftmargin=17pt,
	framexleftmargin=17pt,
	framexrightmargin=17pt,
	framexbottommargin=5pt,
	framextopmargin=5pt,
	showstringspaces=false,
	numbers=left,
	backgroundcolor=\color{white},
	numberstyle=\color{black},
	stringstyle=\color{stringColor},
	keywordstyle=\color{keywordsColor}\bfseries,
	rulecolor=\color{black}
}

\DeclareCaptionFormat{listing}{\rule{\dimexpr\textwidth+17pt\relax}{0.4pt}\par\vskip1pt#1#2#3}
\captionsetup[lstlisting]{format=listing,singlelinecheck=false, margin=0pt, font={sf},labelsep=space,labelfont=bf}

\renewcommand{\lstlistingname}{Codebeispiel}
\renewcommand{\lstlistlistingname}{Codeauflistung}
\newcommand{\source}[1]{\caption*{\hfill Source: {#1}} }
\newcolumntype{M}[1]{>{\centering\arraybackslash}m{#1}}