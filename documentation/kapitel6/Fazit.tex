\chapter{Fazit}

In dieser Arbeit wird beschrieben, wie mit Hilfe eines tiefen neuronalen Netzes die relative Pose eines Agenten in einer simulierten Umgebung inferiert werden kann. Mit dem besonderen Fokus auf das Thema Linienverfolgung, entwickeln wir dabei mehrere Ansätze. Zu Gunsten von stabilerem Fahrverhalten testen wir außerdem die Performance von direkter Inferenz eines Steuerbefehls für einen Duckietown Agenten.

\section{Rückblick}

Im Folgenden reflektieren wir über die Ergebnisse während der Trainings- und Testphase. Anschließend gehen wir auf die Erkenntnisse aus der Validierungsphase ein und wie Unterschiede zwischen Test und Validierung zustande kommen können.

\subsection{Training und Test}

Das Schätzen einer zweidimensionalen Pose erwies sich als am Schwierigsten. Obwohl die Metriken während dem Training mit den Daten mit dem geringstem Anteil an Zufallsposen am besten wirkten, zeigte die Integration des Netzes ein sehr schlechtes Fahrverhalten mit 790 Unglücken in 100000 Schritten. Am besten schnitt das Netz mit dem größten Anteil an Zufallsposen im Datensatz ab. Das Netz, welches mit dem kombinierten Datensatz trainiert wurde, lag in der Mitte.

Eine mögliche Erklärung dafür ist die eher geringe Reaktionsgeschwindigkeit und die Tendenz zum Überschwingen des einfachen PD-Reglers. Durch diese kommt der Agent häufig in Situationen, in welchen er sich eher am Fahrbahnrand und abgeneigt von der Orientierung der Straße befindet. Während das Fahren mit Grundwahrheit mit diesen Überschwingern zurecht kam, können Fehler in der Schätzung hier schnell zu einem Überfahren der Fahrbahnmarkierung führen. Schon geringe Fehler in den Schätzungen führten hier möglicherweise zu einer Abweichung der während der Integration erlebten Verteilung der Input-Daten von der Verteilung der Trainingsdaten. Das Netz konnte also von dem erhöhten Anteil an Zufallsposen im Datensatz profitieren, da dieser eher der tatsächlich in der Integration erlebten Situation ähnelte.

Nach dieser Erklärung müsste das Training mit kombiniertem Datensatz eigentlich das zweitbeste Ergebnis liefern, zumal dieses sogar den geringsten durchschnittlichen Fehler aufweist. Wir halten die Erklärung dennoch für plausibel, da die Werte der Ansätze mit kombiniertem Datensatz und Datensatz mit hohem Anteil an Zufallsposen nahe beieinander liegen. 

\subsection{Validierung}

\section{Verbesserungen}
