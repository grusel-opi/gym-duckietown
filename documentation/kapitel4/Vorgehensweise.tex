
\chapter{Vorgehensweise}

\section{Problemstellung}

Gegeben sei ein Kamerabild $I_D$, welches von einem Duckiebot $D$ aufgenommen wurde. Das Bild $I_D$ wird einem Posenschätzer $E$ zur Verfügung gestellt, welcher dann den Abstand $d_D$ sowie die Orientierung $\Theta_D$ des Duckiebots zur rechten 
Fahrbahnmarkierung ermitteln soll. Der Posenschätzer ist hierbei ein tiefes künstliches neuronales Netz, welches die oben genannte Problemstellung lösen soll. Als Lernverfahren wird überwachtes Lernen eingesetzt.
